To select a candidate feature there needs to be a ranking metric. 
Here the Fisher score is introduced. It is designed to indicate
the discriminatory effect of a feature. Let $y$ be 
the class label vector for $n$ timeseries $y \in \{1,2,\dots,c\}$ and 
$\alpha$ be an interval feature. Then let $\mu^{\alpha}$ be the overall mean of elements in $\alpha$ and $\mu^{\alpha}_k, \sigma^{\alpha}_k$ be the mean and standard
deviation of all the $k$-class elements in $\alpha$. 
The Fisher score is defined as :
\[
  \operatorname{FisherScore}(\alpha, y) = 
  \frac{\sum_{k = 1}^c n_k\left(\mu_k^{\alpha} - \mu^{\alpha}\right)^2}
       {\sum_{k = 1}^{c} n_k\left(\sigma_k^{\alpha}\right)^2}
\]
The Fisher score is chosen for its fast computation, but other metrics are also possible.
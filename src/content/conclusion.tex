Time Series Forest (TSF) nad its successor, the Randomized-Supervised
Time Series Forest(r-STSF), show both complementary advances in classifying time series.
The goal is the same: Achieve interpretable, accurate classification in linear time complexity.
They manage to use simple changes to existing algorithms to built decision trees without resorting to overly
complex feature engineering. TSF introduces the core insight of using 
randomly sampled interval values with a custom splitting criterion, which outperforms
previous algorithms built on entropy gain. This simply adjustment enables TSF to 
outperform strong baselines like NN-DTW on a wide range of benchmarks, while preserving the 
key advantage of interpretability via temporal importance curves.

The reliance on purely random feature generation, limits TSF in settings where the 
relevant temporal structure is sparse or subtle %TODO: ref
Precision is sacrificed for robustness %ref TODO:
Also while being conceptually elegant, the temporal importance curve 
is biased toward central indices due to interval frequency effects. TSF does not introduce 
a mechanism for normalizing or filtering out these uninformative peaks.

These issues are addressed in the construction of STSF, by replacing node-level feature randomness
with tree-level interval supervision. A pre-search of the entire time series using a Fisher score-based strategy
is used to identify candidate discriminatory intervals
It also further enhances coverage with 
through multiple time series representation.
These mechanisms increase both precision and efficiency resulting in STSF achieving better accuracy than TSF while being an 
order of magnitude faster on long series. %TODO: ref

Despite these improvements, STSF comes with new costs: 
interpretability suffers and especially t
%TODO: REF DESPERATLY NEEDED

In sum, TSF and STSF show that it is possible to build time series classifiers that combine
competitiveness with interpretability. TSF leverages randomness to ensure robustness and simplicity and 
r-STSF introduces supervision to increase focus and precision. They establish a foundation for interval-based 
methods that are both practical and transparent. Their success suggests that refining how and where
information is extracted within the time series is preferable to increasing
complexity.
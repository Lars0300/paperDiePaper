\par Time series classification is a supervised learning task of 
assigning labels to sequences of observations. There are two 
basic approaches to this work. 

Instance-based methods use regular classifiers that treat each time point
as a feature. An example for that is One-Nearest-Neighbor With Dynamic Time Warping 
(NNDTW), which is robust to the distortion of the time axis and exceptionally difficult to beat. %TODO: ref from the main paper also quoting.
This method however offers little interpretability 
regarding which temporal regions or patterns differentiate the classes.

In contrast feature-based methods use temporal features calculated over time series intervals. %TODO: ref from main paper 
These are called interval features and they capture localized patterns that may be highly 
discriminative, even when the same patterns occur at slightly different times. They provide 
more interpretable models, because they allow to identify specific regions that contribute to classification decisions.
These classification structures are then called decision trees

Previous work has constructed decision trees using class-based measures such as entropy gain, evaluating
numerous candidate splits derived from interval-based features. However, many of these splits exhibit 
similar class-separation ability. To resolve such ambiguities, more refined measures are needed to 
differentiate between equally effective splits. Additionally, there is demand for a classifier 
that is not only accurate and efficient but also capable of revealing which temporal characteristics 
drive classification decisions.

To fulfill this demand TSF introduces a new split criterion called Entrance Gain. 
This measure has been shown to outperform Entropy Gain and also NNDTW algorithms. 
Computational complexity remains efficient and scalable, because of a random sampling
strategy.

TSF also has its shortcomings however. Because of the random sampling strategy of the intervals,
alignment with meaningful temporal patterns is reduced, leading to suboptimal accuracy and interpretability.
To alleviate this problem STSF is introduced. STSF addresses this by generating a fixed pool of 
intervals prior to training, thus leading to higher alignment to discriminative temporal segments.

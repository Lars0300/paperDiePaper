To build now a tree a recursive function is used. In each recursion step a single node of the 
tree is created. Each node holds an entrance gain $E^*$, an entropy gain $\Delta Entropy^*$, the 
interval $(t_{\mathrm{start}}, t_{\mathrm{end}})^*$, a threshold $\tau^*$ and an interval feature $f_*$.

\subsubsection*{Random Sampling Strategy}
The feature space of all possible intervals for a timeseries the length of $M$
is $O(M^2)$. To reduce this feature space an algorithm $SampleIntervals()$ is introduced. % TODO: ref algo
It first randomly samples $\sqrt{M}$ interval sizes. Then for each interval size $w$ it selects
a set of starting points from $\{1, \dots, M-w+1\}$. The size of the set is $\sqrt{M - w + 1}$, so that 
there are fewer starting points for larger interval sizes. For all these starting points $t_{start}$ the point 
$t_{end} = t_{start} + w - 1 $ is added to the set of all intervals, so that the feature space gets reduced to 
$O(\sqrt{M} \cdot \sqrt{M}) = O(M)$.

\begin{algorithm}[H]
\caption{\textsc{SampleIntervals}$(M)$, The function RandSampNoRep(set, samplesize)
randomly selects samplesize elements from set without replacement.}
\begin{algorithmic}[1]
\State $T_{start} \gets \emptyset$, $T_{end} \gets \emptyset$
\State $W \gets \textsc{RandSampNoRep}(\{1, \dots, M\}, \sqrt{M})$
\ForAll{$w \in W$}
    \State $S \gets \textsc{RandSampNoRep}(\{1, \dots, M - w + 1\}, \sqrt{M - w + 1})$
    \ForAll{$t_{start} \in S$}
        \State $T_{start} \gets T_{start} \cup \{t_{start}\}$
        \State $T_{end}  \gets T_{end}  \cup \{t_{start}  + w - 1\}$
    \EndFor
\EndFor
\State \Return $\langle T_1, T_2 \rangle$
\end{algorithmic}
\end{algorithm}

\subsubsection*{Threshold Set Calculation}

To identify suitable split points for each feature type $f_k$, 
TSF constructs a fixed-size set of candidate thresholds. 
Let $f_k (t_{\mathrm{start}}, t_{\mathrm{end}})$ denote the $k$-th interval feature 
computed over 
the interval $(t_{\mathrm{start}}, t_{\mathrm{end}})$ for all training instances at 
the current node. Let $V = \{ f_k^{(i)} (t_{\mathrm{start}}, t_{\mathrm{end}}) \mid i = 1, \dots, N \}$ 
be the set of resulting feature values across all $N$ 
instances.

TSF then assigns a fixed number of $\kappa$ threshold per feature.
This is achieved by first computing the minimum and 
maximum values in $V$, denoted $a = \min V$ and 
$b = \max V$, respectively. The interval $[a, b]$ 
is then divided into $\kappa + 1$ equal-width segments.
The $\kappa$ internal division points are used as 
candidate thresholds:
\begin{equation*}
\tau_j = a + j \cdot \frac{b - a}{\kappa + 1}, \quad j = 1, 2, \dots, \kappa
\end{equation*}
This method ensures that the thresholds are uniformly 
spaced across the feature value range and that their 
number remains constant regardless of data distribution or 
sample size. 

\subsubsection*{Recursion and Termination}
After getting the interval set from the $SampleIntervals()$ algorithm and computing the threshold-set for each feature type,
TSF loops over every possible interval, every possible threshold and every possible interval feature.
It then calculates the entropy and entrance gain for each combination and maximizes the entrance gain and saves the split data.
Afterwards it checks if the entropy gain is zero. This is the termination condition for the recursion and the node
is marked as a leaf. Then the data is split according to the best features, interval and threshold and the 
function is called on the left and right dataset to build the children of the node.
\begin{algorithm}[H]
\caption{\textsc{BuildTSFTree}$(\mathcal{D})$}
\begin{algorithmic}[1]
\State $\langle T_1, T_2 \rangle \gets \textsc{SampleIntervals}(M)$
\State Compute candidate thresholds $\text{Threshold}_k$ for each feature type $k$
\State Initialize $E^* \gets 0$, $\Delta H^* \gets 0$, $t_1^* \gets 0$, $t_2^* \gets 0$, $\tau^* \gets 0$, $f^* \gets \emptyset$
\ForAll{$(t_{start}, t_{end}) \in \langle T_{start}, T_{end} \rangle$}
    \For{$k = 1$ to $K$}
        \ForAll{$\tau \in \text{Threshold}_k$}
            \State Compute $\Delta Entropy$ and $E$ for $f_k(t_{start}, t_{end}) \le \tau$
            \If{$E > E^*$}
                \State $E^* \gets E$, 
				\State $\Delta Entropy^* \gets \Delta Entropy$, 
				\State $t_{start}^* \gets t_{start}$, 
				\State $t_{end}^* \gets t_{end}$, 
				\State $\tau^* \gets \tau$, 
				\State $f^* \gets f_k$
            \EndIf
        \EndFor
    \EndFor
\EndFor
\If{$\Delta Entropy^* = 0$}
    \State \Return leaf node with majority class of $\mathcal{D}$
\EndIf
\State $\mathcal{D}_{\text{left}} \gets \{(X, y) \in \mathcal{D} \mid f^*(X[t_{start}^*:t_{end}^*]) \le \tau^* \}$
\State $\mathcal{D}_{\text{right}} \gets \{(X, y) \in \mathcal{D} \mid f^*(X[t_{start}^*:t_{end}^*]) > \tau^* \}$
\State node.left $\gets$ \Call{BuildTSFTree}{$\mathcal{D}_{\text{left}}$}
\State node.right $\gets$ \Call{BuildTSFTree}{$\mathcal{D}_{\text{right}}$}
\State \Return internal node with $(f^*, t_{start}^*, t_{end}^*, \tau^*, \text{Left}, \text{Right})$
\end{algorithmic}
\end{algorithm}
